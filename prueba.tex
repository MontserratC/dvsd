% Options for packages loaded elsewhere
\PassOptionsToPackage{unicode}{hyperref}
\PassOptionsToPackage{hyphens}{url}
%
\documentclass[
]{article}
\usepackage{amsmath,amssymb}
\usepackage{iftex}
\ifPDFTeX
  \usepackage[T1]{fontenc}
  \usepackage[utf8]{inputenc}
  \usepackage{textcomp} % provide euro and other symbols
\else % if luatex or xetex
  \usepackage{unicode-math} % this also loads fontspec
  \defaultfontfeatures{Scale=MatchLowercase}
  \defaultfontfeatures[\rmfamily]{Ligatures=TeX,Scale=1}
\fi
\usepackage{lmodern}
\ifPDFTeX\else
  % xetex/luatex font selection
\fi
% Use upquote if available, for straight quotes in verbatim environments
\IfFileExists{upquote.sty}{\usepackage{upquote}}{}
\IfFileExists{microtype.sty}{% use microtype if available
  \usepackage[]{microtype}
  \UseMicrotypeSet[protrusion]{basicmath} % disable protrusion for tt fonts
}{}
\makeatletter
\@ifundefined{KOMAClassName}{% if non-KOMA class
  \IfFileExists{parskip.sty}{%
    \usepackage{parskip}
  }{% else
    \setlength{\parindent}{0pt}
    \setlength{\parskip}{6pt plus 2pt minus 1pt}}
}{% if KOMA class
  \KOMAoptions{parskip=half}}
\makeatother
\usepackage{xcolor}
\usepackage[margin=1in]{geometry}
\usepackage{color}
\usepackage{fancyvrb}
\newcommand{\VerbBar}{|}
\newcommand{\VERB}{\Verb[commandchars=\\\{\}]}
\DefineVerbatimEnvironment{Highlighting}{Verbatim}{commandchars=\\\{\}}
% Add ',fontsize=\small' for more characters per line
\usepackage{framed}
\definecolor{shadecolor}{RGB}{248,248,248}
\newenvironment{Shaded}{\begin{snugshade}}{\end{snugshade}}
\newcommand{\AlertTok}[1]{\textcolor[rgb]{0.94,0.16,0.16}{#1}}
\newcommand{\AnnotationTok}[1]{\textcolor[rgb]{0.56,0.35,0.01}{\textbf{\textit{#1}}}}
\newcommand{\AttributeTok}[1]{\textcolor[rgb]{0.13,0.29,0.53}{#1}}
\newcommand{\BaseNTok}[1]{\textcolor[rgb]{0.00,0.00,0.81}{#1}}
\newcommand{\BuiltInTok}[1]{#1}
\newcommand{\CharTok}[1]{\textcolor[rgb]{0.31,0.60,0.02}{#1}}
\newcommand{\CommentTok}[1]{\textcolor[rgb]{0.56,0.35,0.01}{\textit{#1}}}
\newcommand{\CommentVarTok}[1]{\textcolor[rgb]{0.56,0.35,0.01}{\textbf{\textit{#1}}}}
\newcommand{\ConstantTok}[1]{\textcolor[rgb]{0.56,0.35,0.01}{#1}}
\newcommand{\ControlFlowTok}[1]{\textcolor[rgb]{0.13,0.29,0.53}{\textbf{#1}}}
\newcommand{\DataTypeTok}[1]{\textcolor[rgb]{0.13,0.29,0.53}{#1}}
\newcommand{\DecValTok}[1]{\textcolor[rgb]{0.00,0.00,0.81}{#1}}
\newcommand{\DocumentationTok}[1]{\textcolor[rgb]{0.56,0.35,0.01}{\textbf{\textit{#1}}}}
\newcommand{\ErrorTok}[1]{\textcolor[rgb]{0.64,0.00,0.00}{\textbf{#1}}}
\newcommand{\ExtensionTok}[1]{#1}
\newcommand{\FloatTok}[1]{\textcolor[rgb]{0.00,0.00,0.81}{#1}}
\newcommand{\FunctionTok}[1]{\textcolor[rgb]{0.13,0.29,0.53}{\textbf{#1}}}
\newcommand{\ImportTok}[1]{#1}
\newcommand{\InformationTok}[1]{\textcolor[rgb]{0.56,0.35,0.01}{\textbf{\textit{#1}}}}
\newcommand{\KeywordTok}[1]{\textcolor[rgb]{0.13,0.29,0.53}{\textbf{#1}}}
\newcommand{\NormalTok}[1]{#1}
\newcommand{\OperatorTok}[1]{\textcolor[rgb]{0.81,0.36,0.00}{\textbf{#1}}}
\newcommand{\OtherTok}[1]{\textcolor[rgb]{0.56,0.35,0.01}{#1}}
\newcommand{\PreprocessorTok}[1]{\textcolor[rgb]{0.56,0.35,0.01}{\textit{#1}}}
\newcommand{\RegionMarkerTok}[1]{#1}
\newcommand{\SpecialCharTok}[1]{\textcolor[rgb]{0.81,0.36,0.00}{\textbf{#1}}}
\newcommand{\SpecialStringTok}[1]{\textcolor[rgb]{0.31,0.60,0.02}{#1}}
\newcommand{\StringTok}[1]{\textcolor[rgb]{0.31,0.60,0.02}{#1}}
\newcommand{\VariableTok}[1]{\textcolor[rgb]{0.00,0.00,0.00}{#1}}
\newcommand{\VerbatimStringTok}[1]{\textcolor[rgb]{0.31,0.60,0.02}{#1}}
\newcommand{\WarningTok}[1]{\textcolor[rgb]{0.56,0.35,0.01}{\textbf{\textit{#1}}}}
\usepackage{graphicx}
\makeatletter
\def\maxwidth{\ifdim\Gin@nat@width>\linewidth\linewidth\else\Gin@nat@width\fi}
\def\maxheight{\ifdim\Gin@nat@height>\textheight\textheight\else\Gin@nat@height\fi}
\makeatother
% Scale images if necessary, so that they will not overflow the page
% margins by default, and it is still possible to overwrite the defaults
% using explicit options in \includegraphics[width, height, ...]{}
\setkeys{Gin}{width=\maxwidth,height=\maxheight,keepaspectratio}
% Set default figure placement to htbp
\makeatletter
\def\fps@figure{htbp}
\makeatother
\setlength{\emergencystretch}{3em} % prevent overfull lines
\providecommand{\tightlist}{%
  \setlength{\itemsep}{0pt}\setlength{\parskip}{0pt}}
\setcounter{secnumdepth}{-\maxdimen} % remove section numbering
\ifLuaTeX
  \usepackage{selnolig}  % disable illegal ligatures
\fi
\IfFileExists{bookmark.sty}{\usepackage{bookmark}}{\usepackage{hyperref}}
\IfFileExists{xurl.sty}{\usepackage{xurl}}{} % add URL line breaks if available
\urlstyle{same}
\hypersetup{
  pdftitle={Análisis de Diversidad de Especies por Montserrat Cervantes Espinoza},
  hidelinks,
  pdfcreator={LaTeX via pandoc}}

\title{Análisis de Diversidad de Especies por Montserrat Cervantes
Espinoza}
\author{}
\date{\vspace{-2.5em}2024-09-22}

\begin{document}
\maketitle

\hypertarget{instalar-y-cargar-los-paquetes-necesarios}{%
\section{Instalar y cargar los paquetes
necesarios}\label{instalar-y-cargar-los-paquetes-necesarios}}

\begin{Shaded}
\begin{Highlighting}[]
\FunctionTok{library}\NormalTok{(vegan)}
\end{Highlighting}
\end{Shaded}

\begin{verbatim}
## Loading required package: permute
\end{verbatim}

\begin{verbatim}
## Loading required package: lattice
\end{verbatim}

\begin{verbatim}
## This is vegan 2.6-4
\end{verbatim}

\begin{Shaded}
\begin{Highlighting}[]
\FunctionTok{library}\NormalTok{(betapart)}
\FunctionTok{library}\NormalTok{(ggplot2)}
\end{Highlighting}
\end{Shaded}

\hypertarget{establecer-una-semilla-para-reproducibilidad}{%
\subsection{Establecer una semilla para
reproducibilidad}\label{establecer-una-semilla-para-reproducibilidad}}

\begin{Shaded}
\begin{Highlighting}[]
\FunctionTok{set.seed}\NormalTok{(}\DecValTok{123}\NormalTok{)}
\end{Highlighting}
\end{Shaded}

\hypertarget{crear-una-matrix-con-la-siguiente-estructura}{%
\section{Crear una matrix con la siguiente
estructura}\label{crear-una-matrix-con-la-siguiente-estructura}}

\hypertarget{sitio-filas}{%
\subsubsection{sitio = filas}\label{sitio-filas}}

\hypertarget{especies-columnas}{%
\subsubsection{especies = columnas}\label{especies-columnas}}

\begin{Shaded}
\begin{Highlighting}[]
\NormalTok{abundancia }\OtherTok{\textless{}{-}} \FunctionTok{matrix}\NormalTok{(}\FunctionTok{c}\NormalTok{(}\DecValTok{0}\NormalTok{, }\DecValTok{1}\NormalTok{, }\DecValTok{0}\NormalTok{, }\DecValTok{3}\NormalTok{, }\DecValTok{0}\NormalTok{, }\DecValTok{0}\NormalTok{,}
                       \DecValTok{0}\NormalTok{, }\DecValTok{0}\NormalTok{, }\DecValTok{0}\NormalTok{, }\DecValTok{0}\NormalTok{, }\DecValTok{1}\NormalTok{, }\DecValTok{2}\NormalTok{,}
                       \DecValTok{5}\NormalTok{, }\DecValTok{0}\NormalTok{, }\DecValTok{0}\NormalTok{, }\DecValTok{0}\NormalTok{, }\DecValTok{0}\NormalTok{, }\DecValTok{0}\NormalTok{,}
                       \DecValTok{0}\NormalTok{, }\DecValTok{0}\NormalTok{, }\DecValTok{2}\NormalTok{, }\DecValTok{3}\NormalTok{, }\DecValTok{0}\NormalTok{, }\DecValTok{4}\NormalTok{,}
                       \DecValTok{0}\NormalTok{, }\DecValTok{1}\NormalTok{, }\DecValTok{1}\NormalTok{, }\DecValTok{0}\NormalTok{, }\DecValTok{0}\NormalTok{, }\DecValTok{0}\NormalTok{), }
                     \AttributeTok{nrow =} \DecValTok{5}\NormalTok{, }\AttributeTok{byrow =} \ConstantTok{TRUE}\NormalTok{)}
\FunctionTok{rownames}\NormalTok{(abundancia) }\OtherTok{\textless{}{-}} \FunctionTok{paste}\NormalTok{(}\StringTok{"Sitio"}\NormalTok{, }\DecValTok{1}\SpecialCharTok{:}\DecValTok{5}\NormalTok{)}
\FunctionTok{colnames}\NormalTok{(abundancia) }\OtherTok{\textless{}{-}} \FunctionTok{paste}\NormalTok{(}\StringTok{"Especie"}\NormalTok{, }\DecValTok{1}\SpecialCharTok{:}\DecValTok{6}\NormalTok{)}
\end{Highlighting}
\end{Shaded}

\hypertarget{riqueza}{%
\section{Riqueza}\label{riqueza}}

\hypertarget{simplemente-la-suma-de-filas-numero-de-especies}{%
\subsection{simplemente la suma de filas (numero de
especies)}\label{simplemente-la-suma-de-filas-numero-de-especies}}

\begin{Shaded}
\begin{Highlighting}[]
\NormalTok{riqueza }\OtherTok{\textless{}{-}} \FunctionTok{rowSums}\NormalTok{(abundancia }\SpecialCharTok{\textgreater{}} \DecValTok{0}\NormalTok{)}
\FunctionTok{print}\NormalTok{(}\StringTok{"Riqueza de especies por sitio:"}\NormalTok{)}
\end{Highlighting}
\end{Shaded}

\begin{verbatim}
## [1] "Riqueza de especies por sitio:"
\end{verbatim}

\begin{Shaded}
\begin{Highlighting}[]
\FunctionTok{print}\NormalTok{(riqueza)}
\end{Highlighting}
\end{Shaded}

\begin{verbatim}
## Sitio 1 Sitio 2 Sitio 3 Sitio 4 Sitio 5 
##       2       2       1       3       2
\end{verbatim}

\hypertarget{abundancia}{%
\section{Abundancia}\label{abundancia}}

\hypertarget{abundancia-absoluta-suma-de-abundancias}{%
\subsection{Abundancia Absoluta (suma de
abundancias)}\label{abundancia-absoluta-suma-de-abundancias}}

\begin{Shaded}
\begin{Highlighting}[]
\NormalTok{abundancia\_absoluta }\OtherTok{\textless{}{-}} \FunctionTok{rowSums}\NormalTok{(abundancia)}
\FunctionTok{print}\NormalTok{(}\StringTok{"Abundancia absoluta por sitio:"}\NormalTok{)}
\end{Highlighting}
\end{Shaded}

\begin{verbatim}
## [1] "Abundancia absoluta por sitio:"
\end{verbatim}

\begin{Shaded}
\begin{Highlighting}[]
\FunctionTok{print}\NormalTok{(abundancia\_absoluta)}
\end{Highlighting}
\end{Shaded}

\begin{verbatim}
## Sitio 1 Sitio 2 Sitio 3 Sitio 4 Sitio 5 
##       4       3       5       9       2
\end{verbatim}

\hypertarget{abundancia-relativa-suma-abundancias-entre-uno}{%
\subsection{Abundancia Relativa (suma abundancias entre
uno)}\label{abundancia-relativa-suma-abundancias-entre-uno}}

\begin{Shaded}
\begin{Highlighting}[]
\NormalTok{abundancia\_relativa }\OtherTok{\textless{}{-}} \FunctionTok{prop.table}\NormalTok{(abundancia, }\DecValTok{1}\NormalTok{) }\CommentTok{\# Por filas}
\FunctionTok{print}\NormalTok{(}\StringTok{"Abundancia relativa por sitio"}\NormalTok{)}
\end{Highlighting}
\end{Shaded}

\begin{verbatim}
## [1] "Abundancia relativa por sitio"
\end{verbatim}

\begin{Shaded}
\begin{Highlighting}[]
\FunctionTok{print}\NormalTok{(abundancia\_relativa)}
\end{Highlighting}
\end{Shaded}

\begin{verbatim}
##         Especie 1 Especie 2 Especie 3 Especie 4 Especie 5 Especie 6
## Sitio 1         0      0.25 0.0000000 0.7500000 0.0000000 0.0000000
## Sitio 2         0      0.00 0.0000000 0.0000000 0.3333333 0.6666667
## Sitio 3         1      0.00 0.0000000 0.0000000 0.0000000 0.0000000
## Sitio 4         0      0.00 0.2222222 0.3333333 0.0000000 0.4444444
## Sitio 5         0      0.50 0.5000000 0.0000000 0.0000000 0.0000000
\end{verbatim}

\hypertarget{equitatividad}{%
\section{Equitatividad}\label{equitatividad}}

\hypertarget{uxedndice-de-shannon}{%
\subsection{Índice de Shannon}\label{uxedndice-de-shannon}}

\begin{Shaded}
\begin{Highlighting}[]
\NormalTok{equitatividad\_shannon }\OtherTok{\textless{}{-}} \FunctionTok{diversity}\NormalTok{(abundancia)}
\FunctionTok{print}\NormalTok{(}\StringTok{"Índice de equitatividad de Shannon:"}\NormalTok{)}
\end{Highlighting}
\end{Shaded}

\begin{verbatim}
## [1] "Índice de equitatividad de Shannon:"
\end{verbatim}

\begin{Shaded}
\begin{Highlighting}[]
\FunctionTok{print}\NormalTok{(equitatividad\_shannon)}
\end{Highlighting}
\end{Shaded}

\begin{verbatim}
##   Sitio 1   Sitio 2   Sitio 3   Sitio 4   Sitio 5 
## 0.5623351 0.6365142 0.0000000 1.0608569 0.6931472
\end{verbatim}

\hypertarget{nuxfameros-de-hill}{%
\subsection{Números de Hill}\label{nuxfameros-de-hill}}

\hypertarget{recordar}{%
\subsection{recordar}\label{recordar}}

\textbf{q=0} Equivale a la riqueza de especies (S), es decir, al número
total de especies presentes.

\textbf{q=1} Equivale al exponencial del índice de Shannon (exp(H')),
que considera tanto la riqueza como la equitatividad.

\textbf{q=2} Equivale al inverso del índice de Simpson (1/D), que
pondera más las especies dominantes.

\begin{Shaded}
\begin{Highlighting}[]
\NormalTok{hill\_numbers }\OtherTok{\textless{}{-}} \FunctionTok{renyi}\NormalTok{(abundancia, }\FunctionTok{c}\NormalTok{(}\DecValTok{0}\NormalTok{, }\DecValTok{1}\NormalTok{, }\DecValTok{2}\NormalTok{))}
\FunctionTok{print}\NormalTok{(}\StringTok{"Números de Hill (q=0,1,2):"}\NormalTok{)}
\end{Highlighting}
\end{Shaded}

\begin{verbatim}
## [1] "Números de Hill (q=0,1,2):"
\end{verbatim}

\begin{Shaded}
\begin{Highlighting}[]
\FunctionTok{print}\NormalTok{(hill\_numbers)}
\end{Highlighting}
\end{Shaded}

\begin{verbatim}
##                 0         1         2
## Sitio 1 0.6931472 0.5623351 0.4700036
## Sitio 2 0.6931472 0.6365142 0.5877867
## Sitio 3 0.0000000 0.0000000 0.0000000
## Sitio 4 1.0986123 1.0608569 1.0271533
## Sitio 5 0.6931472 0.6931472 0.6931472
\end{verbatim}

\hypertarget{curvas-de-acumulaciuxf3n}{%
\section{Curvas de acumulación}\label{curvas-de-acumulaciuxf3n}}

\begin{Shaded}
\begin{Highlighting}[]
\CommentTok{\#matrix de abundancia}
\NormalTok{abundancia }\OtherTok{\textless{}{-}} \FunctionTok{matrix}\NormalTok{(}\FunctionTok{c}\NormalTok{(}\DecValTok{0}\NormalTok{, }\DecValTok{1}\NormalTok{, }\DecValTok{0}\NormalTok{, }\DecValTok{3}\NormalTok{, }\DecValTok{0}\NormalTok{, }\DecValTok{0}\NormalTok{,}
                       \DecValTok{0}\NormalTok{, }\DecValTok{0}\NormalTok{, }\DecValTok{0}\NormalTok{, }\DecValTok{0}\NormalTok{, }\DecValTok{1}\NormalTok{, }\DecValTok{2}\NormalTok{,}
                       \DecValTok{5}\NormalTok{, }\DecValTok{0}\NormalTok{, }\DecValTok{0}\NormalTok{, }\DecValTok{0}\NormalTok{, }\DecValTok{0}\NormalTok{, }\DecValTok{0}\NormalTok{,}
                       \DecValTok{0}\NormalTok{, }\DecValTok{0}\NormalTok{, }\DecValTok{2}\NormalTok{, }\DecValTok{3}\NormalTok{, }\DecValTok{0}\NormalTok{, }\DecValTok{4}\NormalTok{,}
                       \DecValTok{0}\NormalTok{, }\DecValTok{1}\NormalTok{, }\DecValTok{1}\NormalTok{, }\DecValTok{0}\NormalTok{, }\DecValTok{0}\NormalTok{, }\DecValTok{0}\NormalTok{), }
                     \AttributeTok{nrow =} \DecValTok{5}\NormalTok{, }\AttributeTok{byrow =} \ConstantTok{TRUE}\NormalTok{)}
\FunctionTok{rownames}\NormalTok{(abundancia) }\OtherTok{\textless{}{-}} \FunctionTok{paste}\NormalTok{(}\StringTok{"Sitio"}\NormalTok{, }\DecValTok{1}\SpecialCharTok{:}\DecValTok{5}\NormalTok{)}
\FunctionTok{colnames}\NormalTok{(abundancia) }\OtherTok{\textless{}{-}} \FunctionTok{paste}\NormalTok{(}\StringTok{"Especie"}\NormalTok{, }\DecValTok{1}\SpecialCharTok{:}\DecValTok{6}\NormalTok{)}
\end{Highlighting}
\end{Shaded}

\hypertarget{anuxe1lisis-de-rarefacciuxf3n}{%
\subsection{Análisis de
Rarefacción}\label{anuxe1lisis-de-rarefacciuxf3n}}

\hypertarget{se-enfoca-en-comparar-la-diversidad-en-diferentes-tamauxf1os-de-muestra.}{%
\subsubsection{se enfoca en comparar la diversidad en diferentes tamaños
de
muestra.}\label{se-enfoca-en-comparar-la-diversidad-en-diferentes-tamauxf1os-de-muestra.}}

\begin{Shaded}
\begin{Highlighting}[]
\NormalTok{abn3 }\OtherTok{\textless{}{-}}\NormalTok{ abundancia}
\NormalTok{s }\OtherTok{\textless{}{-}} \FunctionTok{specnumber}\NormalTok{(abn3)  }\CommentTok{\# Riqueza observada por sitio}
\NormalTok{raremax }\OtherTok{\textless{}{-}} \FunctionTok{min}\NormalTok{(}\FunctionTok{rowSums}\NormalTok{(abn3))  }\CommentTok{\# Máximo número de individuos para rarefacción}
\NormalTok{srare }\OtherTok{\textless{}{-}} \FunctionTok{rarefy}\NormalTok{(abn3, raremax, }\AttributeTok{se =} \ConstantTok{FALSE}\NormalTok{)  }\CommentTok{\# Rarefacción}
\NormalTok{dr }\OtherTok{\textless{}{-}} \FunctionTok{drarefy}\NormalTok{(abn3, raremax)  }\CommentTok{\# Rarefacción con desviaciones estándar}
\end{Highlighting}
\end{Shaded}

\hypertarget{preparar-los-datos-para-la-gruxe1fica}{%
\subsubsection{Preparar los datos para la
gráfica}\label{preparar-los-datos-para-la-gruxe1fica}}

\begin{Shaded}
\begin{Highlighting}[]
\NormalTok{rare\_df }\OtherTok{\textless{}{-}} \FunctionTok{data.frame}\NormalTok{(}
  \AttributeTok{Observado =}\NormalTok{ s,}
  \AttributeTok{Rarefaccionado =}\NormalTok{ srare}
\NormalTok{)}
\end{Highlighting}
\end{Shaded}

\hypertarget{gruxe1fica-de-rarefacciuxf3n}{%
\subsection{Gráfica de Rarefacción}\label{gruxe1fica-de-rarefacciuxf3n}}

\hypertarget{gruxe1fica-de-la-curva-de-rarefacciuxf3n}{%
\subsection{Gráfica de la curva de
rarefacción}\label{gruxe1fica-de-la-curva-de-rarefacciuxf3n}}

\includegraphics{prueba_files/figure-latex/unnamed-chunk-11-1.pdf} \#
\textbf{Analisis de diversidad Beta} \#\# matrix de abundancia

\begin{Shaded}
\begin{Highlighting}[]
\NormalTok{abundancia }\OtherTok{\textless{}{-}} \FunctionTok{matrix}\NormalTok{(}\FunctionTok{c}\NormalTok{(}\DecValTok{0}\NormalTok{, }\DecValTok{1}\NormalTok{, }\DecValTok{0}\NormalTok{, }\DecValTok{3}\NormalTok{, }\DecValTok{0}\NormalTok{, }\DecValTok{0}\NormalTok{,}
                       \DecValTok{0}\NormalTok{, }\DecValTok{0}\NormalTok{, }\DecValTok{0}\NormalTok{, }\DecValTok{0}\NormalTok{, }\DecValTok{1}\NormalTok{, }\DecValTok{2}\NormalTok{,}
                       \DecValTok{5}\NormalTok{, }\DecValTok{0}\NormalTok{, }\DecValTok{0}\NormalTok{, }\DecValTok{0}\NormalTok{, }\DecValTok{0}\NormalTok{, }\DecValTok{0}\NormalTok{,}
                       \DecValTok{0}\NormalTok{, }\DecValTok{0}\NormalTok{, }\DecValTok{2}\NormalTok{, }\DecValTok{3}\NormalTok{, }\DecValTok{0}\NormalTok{, }\DecValTok{4}\NormalTok{,}
                       \DecValTok{0}\NormalTok{, }\DecValTok{1}\NormalTok{, }\DecValTok{1}\NormalTok{, }\DecValTok{0}\NormalTok{, }\DecValTok{0}\NormalTok{, }\DecValTok{0}\NormalTok{), }
                     \AttributeTok{nrow =} \DecValTok{5}\NormalTok{, }\AttributeTok{byrow =} \ConstantTok{TRUE}\NormalTok{)}
\FunctionTok{rownames}\NormalTok{(abundancia) }\OtherTok{\textless{}{-}} \FunctionTok{paste}\NormalTok{(}\StringTok{"Sitio"}\NormalTok{, }\DecValTok{1}\SpecialCharTok{:}\DecValTok{5}\NormalTok{)}
\FunctionTok{colnames}\NormalTok{(abundancia) }\OtherTok{\textless{}{-}} \FunctionTok{paste}\NormalTok{(}\StringTok{"Especie"}\NormalTok{, }\DecValTok{1}\SpecialCharTok{:}\DecValTok{6}\NormalTok{)}
\end{Highlighting}
\end{Shaded}

\hypertarget{imprimir-la-matriz-de-abundancia}{%
\subsubsection{Imprimir la matriz de
abundancia}\label{imprimir-la-matriz-de-abundancia}}

\begin{Shaded}
\begin{Highlighting}[]
\FunctionTok{print}\NormalTok{(}\StringTok{"Matriz de Abundancia:"}\NormalTok{)}
\end{Highlighting}
\end{Shaded}

\begin{verbatim}
## [1] "Matriz de Abundancia:"
\end{verbatim}

\begin{Shaded}
\begin{Highlighting}[]
\FunctionTok{print}\NormalTok{(abundancia)}
\end{Highlighting}
\end{Shaded}

\begin{verbatim}
##         Especie 1 Especie 2 Especie 3 Especie 4 Especie 5 Especie 6
## Sitio 1         0         1         0         3         0         0
## Sitio 2         0         0         0         0         1         2
## Sitio 3         5         0         0         0         0         0
## Sitio 4         0         0         2         3         0         4
## Sitio 5         0         1         1         0         0         0
\end{verbatim}

\hypertarget{crear-una-matriz-de-presencia-ausencia}{%
\section{Crear una matriz de
presencia-ausencia}\label{crear-una-matriz-de-presencia-ausencia}}

\hypertarget{importante-considerar-con-que-datos-se-esta-haciendo-la-diversidad-beta-preferentemente-usar-datos-de-presencia-ausencia.}{%
\subsubsection{Importante considerar con que datos se esta haciendo la
diversidad beta preferentemente usar datos de presencia
ausencia.}\label{importante-considerar-con-que-datos-se-esta-haciendo-la-diversidad-beta-preferentemente-usar-datos-de-presencia-ausencia.}}

\hypertarget{convertir-la-matriz-de-presencia-ausencia-en-un-objeto-de-betapart-paquete}{%
\subsubsection{Convertir la matriz de presencia-ausencia en un objeto de
betapart
(paquete)}\label{convertir-la-matriz-de-presencia-ausencia-en-un-objeto-de-betapart-paquete}}

\hypertarget{diferencia-de-riqueza-de-especies-anidamiento-ux3b2sne.}{%
\subsubsection{\texorpdfstring{Diferencia de riqueza de especies
(anidamiento)
\textbf{(βSNE)}.}{Diferencia de riqueza de especies (anidamiento) (βSNE).}}\label{diferencia-de-riqueza-de-especies-anidamiento-ux3b2sne.}}

\hypertarget{recambio-o-reemplazo-de-la-disimilitud-ux3b2sim}{%
\subsubsection{\texorpdfstring{Recambio o reemplazo de la disimilitud
\textbf{(βSIM)}}{Recambio o reemplazo de la disimilitud (βSIM)}}\label{recambio-o-reemplazo-de-la-disimilitud-ux3b2sim}}

\begin{Shaded}
\begin{Highlighting}[]
\CommentTok{\# Gráficas de densidad}
\FunctionTok{x11}\NormalTok{()}
\FunctionTok{par}\NormalTok{(}\AttributeTok{mar =} \FunctionTok{c}\NormalTok{(}\DecValTok{10}\NormalTok{, }\DecValTok{10}\NormalTok{, }\DecValTok{10}\NormalTok{, }\DecValTok{10}\NormalTok{), }\AttributeTok{xpd =} \ConstantTok{NA}\NormalTok{, }\AttributeTok{cex =} \FloatTok{0.5}\NormalTok{)}

\CommentTok{\# Gráfica de la densidad de diversidad beta}
\FunctionTok{plot}\NormalTok{(}\FunctionTok{density}\NormalTok{(dist.s}\SpecialCharTok{$}\NormalTok{beta.SOR), }\AttributeTok{xlim =} \FunctionTok{c}\NormalTok{(}\DecValTok{0}\NormalTok{, }\DecValTok{2}\NormalTok{), }\AttributeTok{ylim =} \FunctionTok{c}\NormalTok{(}\DecValTok{0}\NormalTok{, }\DecValTok{35}\NormalTok{), }
     \AttributeTok{xlab =} \StringTok{"Diversidad Beta"}\NormalTok{, }\AttributeTok{main =} \StringTok{""}\NormalTok{, }\AttributeTok{lwd =} \DecValTok{3}\NormalTok{)}
\FunctionTok{lines}\NormalTok{(}\FunctionTok{density}\NormalTok{(dist.s}\SpecialCharTok{$}\NormalTok{beta.SNE), }\AttributeTok{lty =} \DecValTok{1}\NormalTok{, }\AttributeTok{lwd =} \DecValTok{2}\NormalTok{, }\AttributeTok{col =} \StringTok{"red"}\NormalTok{)}
\FunctionTok{lines}\NormalTok{(}\FunctionTok{density}\NormalTok{(dist.s}\SpecialCharTok{$}\NormalTok{beta.SIM), }\AttributeTok{lty =} \DecValTok{2}\NormalTok{, }\AttributeTok{lwd =} \DecValTok{2}\NormalTok{, }\AttributeTok{col =} \StringTok{"grey60"}\NormalTok{)}
\end{Highlighting}
\end{Shaded}

\includegraphics{prueba_files/figure-latex/unnamed-chunk-16-1.pdf}

\begin{Shaded}
\begin{Highlighting}[]
\CommentTok{\# Gráfica alternativa}
\FunctionTok{plot}\NormalTok{(}\FunctionTok{density}\NormalTok{(dist.s}\SpecialCharTok{$}\NormalTok{beta.SOR), }\AttributeTok{xlim =} \FunctionTok{c}\NormalTok{(}\DecValTok{0}\NormalTok{, }\FloatTok{0.8}\NormalTok{), }\AttributeTok{ylim =} \FunctionTok{c}\NormalTok{(}\DecValTok{0}\NormalTok{, }\DecValTok{19}\NormalTok{), }
     \AttributeTok{xlab =} \StringTok{"Diversidad Beta"}\NormalTok{, }\AttributeTok{main =} \StringTok{""}\NormalTok{, }\AttributeTok{lwd =} \DecValTok{3}\NormalTok{)}
\FunctionTok{lines}\NormalTok{(}\FunctionTok{density}\NormalTok{(dist.s}\SpecialCharTok{$}\NormalTok{beta.SNE), }\AttributeTok{lty =} \DecValTok{1}\NormalTok{, }\AttributeTok{lwd =} \DecValTok{2}\NormalTok{)}
\FunctionTok{lines}\NormalTok{(}\FunctionTok{density}\NormalTok{(dist.s}\SpecialCharTok{$}\NormalTok{beta.SIM), }\AttributeTok{lty =} \DecValTok{2}\NormalTok{, }\AttributeTok{lwd =} \DecValTok{2}\NormalTok{)}
\end{Highlighting}
\end{Shaded}

\includegraphics{prueba_files/figure-latex/unnamed-chunk-16-2.pdf}

\hypertarget{abundancia-por-sitio-comparacion-simple-analisis-exploratorio}{%
\section{Abundancia por sitio comparacion simple analisis
exploratorio}\label{abundancia-por-sitio-comparacion-simple-analisis-exploratorio}}

\begin{Shaded}
\begin{Highlighting}[]
\NormalTok{abundancia }\OtherTok{\textless{}{-}} \FunctionTok{matrix}\NormalTok{(}\FunctionTok{c}\NormalTok{(}\DecValTok{0}\NormalTok{, }\DecValTok{1}\NormalTok{, }\DecValTok{0}\NormalTok{, }\DecValTok{3}\NormalTok{, }\DecValTok{0}\NormalTok{, }\DecValTok{0}\NormalTok{,}
                       \DecValTok{0}\NormalTok{, }\DecValTok{0}\NormalTok{, }\DecValTok{0}\NormalTok{, }\DecValTok{0}\NormalTok{, }\DecValTok{1}\NormalTok{, }\DecValTok{2}\NormalTok{,}
                       \DecValTok{5}\NormalTok{, }\DecValTok{0}\NormalTok{, }\DecValTok{0}\NormalTok{, }\DecValTok{0}\NormalTok{, }\DecValTok{0}\NormalTok{, }\DecValTok{0}\NormalTok{,}
                       \DecValTok{0}\NormalTok{, }\DecValTok{0}\NormalTok{, }\DecValTok{2}\NormalTok{, }\DecValTok{3}\NormalTok{, }\DecValTok{0}\NormalTok{, }\DecValTok{4}\NormalTok{,}
                       \DecValTok{0}\NormalTok{, }\DecValTok{1}\NormalTok{, }\DecValTok{1}\NormalTok{, }\DecValTok{0}\NormalTok{, }\DecValTok{0}\NormalTok{, }\DecValTok{0}\NormalTok{), }
                     \AttributeTok{nrow =} \DecValTok{5}\NormalTok{, }\AttributeTok{byrow =} \ConstantTok{TRUE}\NormalTok{)}
\FunctionTok{rownames}\NormalTok{(abundancia) }\OtherTok{\textless{}{-}} \FunctionTok{paste}\NormalTok{(}\StringTok{"Sitio"}\NormalTok{, }\DecValTok{1}\SpecialCharTok{:}\DecValTok{5}\NormalTok{)}
\FunctionTok{colnames}\NormalTok{(abundancia) }\OtherTok{\textless{}{-}} \FunctionTok{paste}\NormalTok{(}\StringTok{"Especie"}\NormalTok{, }\DecValTok{1}\SpecialCharTok{:}\DecValTok{6}\NormalTok{)}

\CommentTok{\# Calcular la abundancia total por sitio}
\NormalTok{abundancia\_total }\OtherTok{\textless{}{-}} \FunctionTok{rowSums}\NormalTok{(abundancia)}
\NormalTok{abundancia\_grafico }\OtherTok{\textless{}{-}} \FunctionTok{data.frame}\NormalTok{(}\AttributeTok{Sitio =} \FunctionTok{rownames}\NormalTok{(abundancia), }\AttributeTok{Abundancia =}\NormalTok{ abundancia\_total)}

\CommentTok{\# Calcular la desviación estándar y error}
\NormalTok{abundancia\_sd }\OtherTok{\textless{}{-}} \FunctionTok{apply}\NormalTok{(abundancia, }\DecValTok{1}\NormalTok{, sd)  }\CommentTok{\# Desviación estándar para cada sitio}
\NormalTok{n }\OtherTok{\textless{}{-}} \FunctionTok{ncol}\NormalTok{(abundancia)  }\CommentTok{\# Número de especies}
\NormalTok{error }\OtherTok{\textless{}{-}} \FunctionTok{qnorm}\NormalTok{(}\FloatTok{0.975}\NormalTok{) }\SpecialCharTok{*}\NormalTok{ (abundancia\_sd }\SpecialCharTok{/} \FunctionTok{sqrt}\NormalTok{(n))  }\CommentTok{\# Error estándar para el IC del 95\%}
\end{Highlighting}
\end{Shaded}

\begin{Shaded}
\begin{Highlighting}[]
\FunctionTok{ggplot}\NormalTok{(abundancia\_grafico, }\FunctionTok{aes}\NormalTok{(}\AttributeTok{x =}\NormalTok{ Sitio, }\AttributeTok{y =}\NormalTok{ Abundancia)) }\SpecialCharTok{+}
  \FunctionTok{geom\_bar}\NormalTok{(}\AttributeTok{stat =} \StringTok{"identity"}\NormalTok{, }\AttributeTok{fill =} \StringTok{"lightgreen"}\NormalTok{) }\SpecialCharTok{+}
  \FunctionTok{geom\_errorbar}\NormalTok{(}\FunctionTok{aes}\NormalTok{(}\AttributeTok{ymin =}\NormalTok{ Abundancia }\SpecialCharTok{{-}}\NormalTok{ error, }\AttributeTok{ymax =}\NormalTok{ Abundancia }\SpecialCharTok{+}\NormalTok{ error), }\AttributeTok{width =} \FloatTok{0.2}\NormalTok{, }\AttributeTok{color =} \StringTok{"red"}\NormalTok{) }\SpecialCharTok{+}
  \FunctionTok{labs}\NormalTok{(}\AttributeTok{title =} \StringTok{"Abundancia por sitio"}\NormalTok{, }\AttributeTok{y =} \StringTok{"Abundancia"}\NormalTok{, }\AttributeTok{x =} \StringTok{"Sitios"}\NormalTok{) }\SpecialCharTok{+}
  \FunctionTok{theme\_minimal}\NormalTok{()}
\end{Highlighting}
\end{Shaded}

\includegraphics{prueba_files/figure-latex/unnamed-chunk-18-1.pdf} \# 5.
Análisis de la Diversidad Beta con vegan

\begin{Shaded}
\begin{Highlighting}[]
\NormalTok{beta\_diversidad }\OtherTok{\textless{}{-}} \FunctionTok{vegdist}\NormalTok{(abundancia)}
\end{Highlighting}
\end{Shaded}

\hypertarget{descomposiciuxf3n-de-la-beta-diversidad}{%
\section{Descomposición de la Beta
Diversidad}\label{descomposiciuxf3n-de-la-beta-diversidad}}

\begin{Shaded}
\begin{Highlighting}[]
\NormalTok{anidamiento\_recambio }\OtherTok{\textless{}{-}} \FunctionTok{betadisper}\NormalTok{(beta\_diversidad, }\AttributeTok{group =} \FunctionTok{rownames}\NormalTok{(abundancia))}
\end{Highlighting}
\end{Shaded}

\begin{Shaded}
\begin{Highlighting}[]
\FunctionTok{library}\NormalTok{(vegan)}
\FunctionTok{library}\NormalTok{(ggplot2)}
\FunctionTok{library}\NormalTok{(reshape2)}

\NormalTok{abundancia }\OtherTok{\textless{}{-}} \FunctionTok{matrix}\NormalTok{(}\FunctionTok{c}\NormalTok{(}\DecValTok{5}\NormalTok{, }\DecValTok{3}\NormalTok{, }\DecValTok{0}\NormalTok{, }\DecValTok{2}\NormalTok{,}
                       \DecValTok{0}\NormalTok{, }\DecValTok{1}\NormalTok{, }\DecValTok{4}\NormalTok{, }\DecValTok{3}\NormalTok{,}
                       \DecValTok{2}\NormalTok{, }\DecValTok{0}\NormalTok{, }\DecValTok{5}\NormalTok{, }\DecValTok{1}\NormalTok{,}
                       \DecValTok{1}\NormalTok{, }\DecValTok{3}\NormalTok{, }\DecValTok{0}\NormalTok{, }\DecValTok{0}\NormalTok{),}
                     \AttributeTok{nrow =} \DecValTok{4}\NormalTok{, }\AttributeTok{byrow =} \ConstantTok{TRUE}\NormalTok{)}
\FunctionTok{rownames}\NormalTok{(abundancia) }\OtherTok{\textless{}{-}} \FunctionTok{paste}\NormalTok{(}\StringTok{"Sitio"}\NormalTok{, }\DecValTok{1}\SpecialCharTok{:}\DecValTok{4}\NormalTok{)}
\FunctionTok{colnames}\NormalTok{(abundancia) }\OtherTok{\textless{}{-}} \FunctionTok{paste}\NormalTok{(}\StringTok{"Bigote"}\NormalTok{, }\DecValTok{1}\SpecialCharTok{:}\DecValTok{4}\NormalTok{)}

\CommentTok{\# Convertir la matriz de abundancia a presencia/ausencia}
\NormalTok{presencia\_ausencia }\OtherTok{\textless{}{-}} \FunctionTok{ifelse}\NormalTok{(abundancia }\SpecialCharTok{\textgreater{}} \DecValTok{0}\NormalTok{, }\DecValTok{1}\NormalTok{, }\DecValTok{0}\NormalTok{)}

\CommentTok{\# Calcular la diversidad beta usando la distancia de Jaccard}
\NormalTok{beta\_diversidad }\OtherTok{\textless{}{-}} \FunctionTok{vegdist}\NormalTok{(presencia\_ausencia, }\AttributeTok{method =} \StringTok{"jaccard"}\NormalTok{)}

\CommentTok{\# Convertir a data frame}
\NormalTok{beta\_df }\OtherTok{\textless{}{-}} \FunctionTok{as.data.frame}\NormalTok{(}\FunctionTok{as.matrix}\NormalTok{(beta\_diversidad))}

\CommentTok{\# Agregar los nombres de los sitios como columnas}
\NormalTok{beta\_df}\SpecialCharTok{$}\NormalTok{Sitio1 }\OtherTok{\textless{}{-}} \FunctionTok{rownames}\NormalTok{(beta\_df)}

\CommentTok{\# Transformar a formato largo}
\NormalTok{beta\_long }\OtherTok{\textless{}{-}} \FunctionTok{melt}\NormalTok{(beta\_df, }\AttributeTok{id.vars =} \StringTok{"Sitio1"}\NormalTok{, }\AttributeTok{variable.name =} \StringTok{"Sitio2"}\NormalTok{, }\AttributeTok{value.name =} \StringTok{"Diversidad\_Beta"}\NormalTok{)}

\CommentTok{\# Filtrar para evitar comparaciones consigo mismo}
\NormalTok{beta\_long }\OtherTok{\textless{}{-}}\NormalTok{ beta\_long[beta\_long}\SpecialCharTok{$}\NormalTok{Sitio1 }\SpecialCharTok{!=}\NormalTok{ beta\_long}\SpecialCharTok{$}\NormalTok{Sitio2, ]}

\CommentTok{\# Graficar el boxplot de la diversidad beta}
\FunctionTok{ggplot}\NormalTok{(beta\_long, }\FunctionTok{aes}\NormalTok{(}\AttributeTok{x =}\NormalTok{ Sitio1, }\AttributeTok{y =}\NormalTok{ Diversidad\_Beta)) }\SpecialCharTok{+}
  \FunctionTok{geom\_boxplot}\NormalTok{(}\AttributeTok{fill =} \StringTok{"skyblue"}\NormalTok{, }\AttributeTok{outlier.color =} \StringTok{"red"}\NormalTok{) }\SpecialCharTok{+}
  \FunctionTok{stat\_summary}\NormalTok{(}\AttributeTok{fun =}\NormalTok{ mean, }\AttributeTok{geom =} \StringTok{"point"}\NormalTok{, }\AttributeTok{shape =} \DecValTok{18}\NormalTok{, }\AttributeTok{size =} \DecValTok{4}\NormalTok{, }\AttributeTok{color =} \StringTok{"darkblue"}\NormalTok{, }
               \AttributeTok{position =} \FunctionTok{position\_dodge}\NormalTok{(}\FloatTok{0.75}\NormalTok{)) }\SpecialCharTok{+}  \CommentTok{\# Promedio}
  \FunctionTok{stat\_summary}\NormalTok{(}\AttributeTok{fun.data =}\NormalTok{ mean\_cl\_normal, }\AttributeTok{geom =} \StringTok{"errorbar"}\NormalTok{, }\AttributeTok{width =} \FloatTok{0.2}\NormalTok{, }
               \AttributeTok{color =} \StringTok{"darkblue"}\NormalTok{, }\AttributeTok{position =} \FunctionTok{position\_dodge}\NormalTok{(}\FloatTok{0.75}\NormalTok{)) }\SpecialCharTok{+}  \CommentTok{\# Intervalos de confianza}
  \FunctionTok{labs}\NormalTok{(}\AttributeTok{title =} \StringTok{"Boxplot de Diversidad Beta"}\NormalTok{,}
       \AttributeTok{x =} \StringTok{"Sitios"}\NormalTok{,}
       \AttributeTok{y =} \StringTok{"Diversidad Beta (Jaccard)"}\NormalTok{) }\SpecialCharTok{+}
  \FunctionTok{theme\_minimal}\NormalTok{() }\SpecialCharTok{+}
  \FunctionTok{theme}\NormalTok{(}\AttributeTok{axis.text.x =} \FunctionTok{element\_text}\NormalTok{(}\AttributeTok{angle =} \DecValTok{45}\NormalTok{, }\AttributeTok{hjust =} \DecValTok{1}\NormalTok{))}
\end{Highlighting}
\end{Shaded}

\includegraphics{prueba_files/figure-latex/unnamed-chunk-21-1.pdf}

\begin{Shaded}
\begin{Highlighting}[]
\NormalTok{data }\OtherTok{\textless{}{-}} \FunctionTok{matrix}\NormalTok{(}\FunctionTok{c}\NormalTok{(}\DecValTok{1}\NormalTok{, }\DecValTok{0}\NormalTok{, }\DecValTok{1}\NormalTok{, }\DecValTok{0}\NormalTok{, }
                 \DecValTok{1}\NormalTok{, }\DecValTok{1}\NormalTok{, }\DecValTok{0}\NormalTok{, }\DecValTok{1}\NormalTok{, }
                 \DecValTok{0}\NormalTok{, }\DecValTok{1}\NormalTok{, }\DecValTok{1}\NormalTok{, }\DecValTok{0}\NormalTok{, }
                 \DecValTok{1}\NormalTok{, }\DecValTok{0}\NormalTok{, }\DecValTok{0}\NormalTok{, }\DecValTok{1}\NormalTok{), }
               \AttributeTok{nrow =} \DecValTok{4}\NormalTok{, }\AttributeTok{byrow =} \ConstantTok{TRUE}\NormalTok{)}

\CommentTok{\# Calcular los componentes de beta diversidad}
\NormalTok{beta }\OtherTok{\textless{}{-}} \FunctionTok{beta.pair}\NormalTok{(data, }\AttributeTok{index.family =} \StringTok{"jaccard"}\NormalTok{)}

\CommentTok{\# Resultados}
\NormalTok{beta.sim }\OtherTok{\textless{}{-}}\NormalTok{ beta}\SpecialCharTok{$}\NormalTok{beta.sim  }\CommentTok{\# Reemplazo de especies}
\NormalTok{beta.sne }\OtherTok{\textless{}{-}}\NormalTok{ beta}\SpecialCharTok{$}\NormalTok{beta.sne  }\CommentTok{\# Diferencia en riqueza de especies}
\NormalTok{beta.sor }\OtherTok{\textless{}{-}}\NormalTok{ beta}\SpecialCharTok{$}\NormalTok{beta.sor  }\CommentTok{\# Diversidad beta total}

\CommentTok{\# Crear un dataframe con los resultados de beta diversidad}
\NormalTok{df }\OtherTok{\textless{}{-}} \FunctionTok{data.frame}\NormalTok{(}
  \AttributeTok{Grupo =} \FunctionTok{c}\NormalTok{(}\StringTok{"anfibios"}\NormalTok{, }\StringTok{"reptiles"}\NormalTok{, }\StringTok{"mamíferos"}\NormalTok{, }\StringTok{"aves"}\NormalTok{),}
  \AttributeTok{Reemplazo =} \FunctionTok{c}\NormalTok{(}\FloatTok{0.7}\NormalTok{, }\FloatTok{0.5}\NormalTok{, }\FloatTok{0.6}\NormalTok{, }\FloatTok{0.2}\NormalTok{), }\CommentTok{\# Valores simulados, reemplázalos con tus datos}
  \AttributeTok{Riqueza =} \FunctionTok{c}\NormalTok{(}\DecValTok{0}\NormalTok{, }\DecValTok{0}\NormalTok{, }\DecValTok{0}\NormalTok{, }\FloatTok{0.3}\NormalTok{)          }\CommentTok{\# Valores simulados, reemplázalos con tus datos}
\NormalTok{)}

\CommentTok{\# Instala ggplot2 si no lo tienes}
\ControlFlowTok{if}\NormalTok{(}\SpecialCharTok{!}\FunctionTok{require}\NormalTok{(ggplot2)) }\FunctionTok{install.packages}\NormalTok{(}\StringTok{"ggplot2"}\NormalTok{)}

\CommentTok{\# Cargar ggplot2}
\FunctionTok{library}\NormalTok{(ggplot2)}

\CommentTok{\# Crear la gráfica}
\FunctionTok{ggplot}\NormalTok{(df, }\FunctionTok{aes}\NormalTok{(}\AttributeTok{x =}\NormalTok{ Grupo)) }\SpecialCharTok{+}
  \FunctionTok{geom\_bar}\NormalTok{(}\FunctionTok{aes}\NormalTok{(}\AttributeTok{y =}\NormalTok{ Reemplazo }\SpecialCharTok{+}\NormalTok{ Riqueza, }\AttributeTok{fill =} \StringTok{"Anidamiento"}\NormalTok{), }\AttributeTok{stat =} \StringTok{"identity"}\NormalTok{, }\AttributeTok{color =} \StringTok{"black"}\NormalTok{) }\SpecialCharTok{+}
  \FunctionTok{geom\_bar}\NormalTok{(}\FunctionTok{aes}\NormalTok{(}\AttributeTok{y =}\NormalTok{ Reemplazo, }\AttributeTok{fill =} \StringTok{"Reemplazo"}\NormalTok{), }\AttributeTok{stat =} \StringTok{"identity"}\NormalTok{, }\AttributeTok{color =} \StringTok{"black"}\NormalTok{) }\SpecialCharTok{+}
  \FunctionTok{scale\_fill\_manual}\NormalTok{(}\AttributeTok{name =} \StringTok{"Componentes"}\NormalTok{, }\AttributeTok{values =} \FunctionTok{c}\NormalTok{(}\StringTok{"Reemplazo"} \OtherTok{=} \StringTok{"gray"}\NormalTok{, }\StringTok{"Anidamiento"} \OtherTok{=} \StringTok{"purple"}\NormalTok{)) }\SpecialCharTok{+}
  \FunctionTok{labs}\NormalTok{(}\AttributeTok{y =} \StringTok{"Diferencias en composición de especies"}\NormalTok{, }\AttributeTok{x =} \StringTok{""}\NormalTok{, }
       \AttributeTok{title =} \StringTok{"Diferencias en composición de especies por grupo o gremio"}\NormalTok{) }\SpecialCharTok{+}
  \FunctionTok{theme\_minimal}\NormalTok{()}
\end{Highlighting}
\end{Shaded}

\includegraphics{prueba_files/figure-latex/unnamed-chunk-22-1.pdf}

\hypertarget{anuxe1lisis-de-la-diversidad-beta-nmds}{%
\section{Análisis de la Diversidad Beta
NMDS}\label{anuxe1lisis-de-la-diversidad-beta-nmds}}

\begin{Shaded}
\begin{Highlighting}[]
\NormalTok{abundancia }\OtherTok{\textless{}{-}} \FunctionTok{matrix}\NormalTok{(}\FunctionTok{c}\NormalTok{(}\DecValTok{5}\NormalTok{, }\DecValTok{0}\NormalTok{, }\DecValTok{3}\NormalTok{, }\DecValTok{0}\NormalTok{, }\DecValTok{0}\NormalTok{, }\DecValTok{2}\NormalTok{, }\DecValTok{5}\NormalTok{, }\DecValTok{1}\NormalTok{, }\DecValTok{0}\NormalTok{, }\DecValTok{0}\NormalTok{, }\DecValTok{4}\NormalTok{, }\DecValTok{2}\NormalTok{, }\DecValTok{1}\NormalTok{, }\DecValTok{0}\NormalTok{, }\DecValTok{3}\NormalTok{, }\DecValTok{6}\NormalTok{), }
                     \AttributeTok{nrow =} \DecValTok{4}\NormalTok{, }\AttributeTok{byrow =} \ConstantTok{TRUE}\NormalTok{)}
\FunctionTok{rownames}\NormalTok{(abundancia) }\OtherTok{\textless{}{-}} \FunctionTok{c}\NormalTok{(}\StringTok{"Sitio1"}\NormalTok{, }\StringTok{"Sitio2"}\NormalTok{, }\StringTok{"Sitio3"}\NormalTok{, }\StringTok{"Sitio4"}\NormalTok{)}
\end{Highlighting}
\end{Shaded}

\begin{Shaded}
\begin{Highlighting}[]
\CommentTok{\# Graficar el NMDS con los puntos adicionales}
\FunctionTok{ggplot}\NormalTok{(nmds\_plot, }\FunctionTok{aes}\NormalTok{(}\AttributeTok{x =}\NormalTok{ MDS1, }\AttributeTok{y =}\NormalTok{ MDS2, }\AttributeTok{shape =}\NormalTok{ Mes, }\AttributeTok{color =}\NormalTok{ Hora)) }\SpecialCharTok{+}
  \FunctionTok{geom\_point}\NormalTok{(}\AttributeTok{size =} \DecValTok{4}\NormalTok{) }\SpecialCharTok{+} 
  \FunctionTok{stat\_ellipse}\NormalTok{(}\FunctionTok{aes}\NormalTok{(}\AttributeTok{group =}\NormalTok{ Mes), }\AttributeTok{linetype =} \StringTok{"dotted"}\NormalTok{) }\SpecialCharTok{+}  \CommentTok{\# Elipses por grupo (Mes)}
  \FunctionTok{scale\_shape\_manual}\NormalTok{(}\AttributeTok{values =} \FunctionTok{c}\NormalTok{(}\DecValTok{21}\NormalTok{, }\DecValTok{22}\NormalTok{, }\DecValTok{23}\NormalTok{, }\DecValTok{24}\NormalTok{, }\DecValTok{25}\NormalTok{)) }\SpecialCharTok{+}  \CommentTok{\# Diferentes formas para los meses}
  \FunctionTok{scale\_color\_manual}\NormalTok{(}\AttributeTok{values =} \FunctionTok{c}\NormalTok{(}\StringTok{"black"}\NormalTok{, }\StringTok{"gray"}\NormalTok{, }\StringTok{"blue"}\NormalTok{, }\StringTok{"red"}\NormalTok{)) }\SpecialCharTok{+}  \CommentTok{\# Colores para las horas}
  \FunctionTok{labs}\NormalTok{(}\AttributeTok{title =} \StringTok{"Escalado Multidimensional No{-}Métrico (NMDS)"}\NormalTok{, }
       \AttributeTok{x =} \StringTok{"Dimensión 1"}\NormalTok{, }\AttributeTok{y =} \StringTok{"Dimensión 2"}\NormalTok{) }\SpecialCharTok{+}
  \FunctionTok{theme\_minimal}\NormalTok{() }\SpecialCharTok{+}
  \FunctionTok{theme}\NormalTok{(}\AttributeTok{legend.position =} \StringTok{"right"}\NormalTok{)}
\end{Highlighting}
\end{Shaded}

\includegraphics{prueba_files/figure-latex/unnamed-chunk-25-1.pdf}

\hypertarget{gruxe1fica-de-pca-con-ejes-y-vectores}{%
\section{Gráfica de PCA con ejes y
vectores}\label{gruxe1fica-de-pca-con-ejes-y-vectores}}

\begin{Shaded}
\begin{Highlighting}[]
\CommentTok{\# Análisis de Coordenadas Principales (PCA)}
\FunctionTok{set.seed}\NormalTok{(}\DecValTok{123}\NormalTok{)}
\NormalTok{abundancia }\OtherTok{\textless{}{-}} \FunctionTok{matrix}\NormalTok{(}\FunctionTok{sample}\NormalTok{(}\DecValTok{1}\SpecialCharTok{:}\DecValTok{20}\NormalTok{, }\DecValTok{30} \SpecialCharTok{*} \DecValTok{5}\NormalTok{, }\AttributeTok{replace =} \ConstantTok{TRUE}\NormalTok{), }\AttributeTok{ncol =} \DecValTok{5}\NormalTok{)}
\FunctionTok{rownames}\NormalTok{(abundancia) }\OtherTok{\textless{}{-}} \FunctionTok{paste}\NormalTok{(}\StringTok{"Sitio"}\NormalTok{, }\DecValTok{1}\SpecialCharTok{:}\DecValTok{30}\NormalTok{, }\AttributeTok{sep =} \StringTok{""}\NormalTok{)}
\FunctionTok{colnames}\NormalTok{(abundancia) }\OtherTok{\textless{}{-}} \FunctionTok{paste}\NormalTok{(}\StringTok{"Especie"}\NormalTok{, }\DecValTok{1}\SpecialCharTok{:}\DecValTok{5}\NormalTok{, }\AttributeTok{sep =} \StringTok{""}\NormalTok{)}

\CommentTok{\# Añadir una variable para los grupos}
\NormalTok{grupos }\OtherTok{\textless{}{-}} \FunctionTok{factor}\NormalTok{(}\FunctionTok{rep}\NormalTok{(}\FunctionTok{c}\NormalTok{(}\StringTok{"Cardonal"}\NormalTok{, }\StringTok{"Espinar"}\NormalTok{, }\StringTok{"Matorral desértico"}\NormalTok{), }\AttributeTok{each =} \DecValTok{10}\NormalTok{))}

\CommentTok{\# Cálculo de disimilitud Bray{-}Curtis y el PCA (o escalado multidimensional)}
\NormalTok{disimilitud }\OtherTok{\textless{}{-}} \FunctionTok{vegdist}\NormalTok{(abundancia, }\AttributeTok{method =} \StringTok{"bray"}\NormalTok{)}
\NormalTok{pca\_resultados }\OtherTok{\textless{}{-}} \FunctionTok{cmdscale}\NormalTok{(disimilitud, }\AttributeTok{eig =} \ConstantTok{TRUE}\NormalTok{)}

\CommentTok{\# Convertir los resultados en un data frame}
\NormalTok{pca\_points }\OtherTok{\textless{}{-}} \FunctionTok{data.frame}\NormalTok{(pca\_resultados}\SpecialCharTok{$}\NormalTok{points)}
\FunctionTok{colnames}\NormalTok{(pca\_points) }\OtherTok{\textless{}{-}} \FunctionTok{c}\NormalTok{(}\StringTok{"PC1"}\NormalTok{, }\StringTok{"PC2"}\NormalTok{)}
\NormalTok{pca\_points}\SpecialCharTok{$}\NormalTok{Sitio }\OtherTok{\textless{}{-}} \FunctionTok{rownames}\NormalTok{(abundancia)}
\NormalTok{pca\_points}\SpecialCharTok{$}\NormalTok{Grupo }\OtherTok{\textless{}{-}}\NormalTok{ grupos}

\CommentTok{\# Calcular los centroides para cada grupo}
\NormalTok{centroides }\OtherTok{\textless{}{-}} \FunctionTok{aggregate}\NormalTok{(}\FunctionTok{cbind}\NormalTok{(PC1, PC2) }\SpecialCharTok{\textasciitilde{}}\NormalTok{ Grupo, }\AttributeTok{data =}\NormalTok{ pca\_points, }\AttributeTok{FUN =}\NormalTok{ mean)}

\CommentTok{\# Combinar los puntos con los centroides}
\NormalTok{pca\_combined }\OtherTok{\textless{}{-}} \FunctionTok{merge}\NormalTok{(pca\_points, centroides, }\AttributeTok{by =} \StringTok{"Grupo"}\NormalTok{, }\AttributeTok{suffixes =} \FunctionTok{c}\NormalTok{(}\StringTok{""}\NormalTok{, }\StringTok{"\_centroide"}\NormalTok{))}

\CommentTok{\# Graficar el PCA con puntos, centroides y elipses}
\FunctionTok{ggplot}\NormalTok{(pca\_combined, }\FunctionTok{aes}\NormalTok{(}\AttributeTok{x =}\NormalTok{ PC1, }\AttributeTok{y =}\NormalTok{ PC2, }\AttributeTok{color =}\NormalTok{ Grupo, }\AttributeTok{shape =}\NormalTok{ Grupo)) }\SpecialCharTok{+}
  \FunctionTok{geom\_point}\NormalTok{(}\AttributeTok{size =} \DecValTok{3}\NormalTok{) }\SpecialCharTok{+}  \CommentTok{\# Puntos por grupo}
  \FunctionTok{geom\_point}\NormalTok{(}\FunctionTok{aes}\NormalTok{(}\AttributeTok{x =}\NormalTok{ PC1\_centroide, }\AttributeTok{y =}\NormalTok{ PC2\_centroide), }\AttributeTok{shape =} \DecValTok{4}\NormalTok{, }\AttributeTok{size =} \DecValTok{4}\NormalTok{, }\AttributeTok{color =} \StringTok{"red"}\NormalTok{) }\SpecialCharTok{+}  \CommentTok{\# Centroides}
  \FunctionTok{geom\_segment}\NormalTok{(}\FunctionTok{aes}\NormalTok{(}\AttributeTok{xend =}\NormalTok{ PC1\_centroide, }\AttributeTok{yend =}\NormalTok{ PC2\_centroide), }\AttributeTok{linetype =} \StringTok{"dotted"}\NormalTok{, }\AttributeTok{color =} \StringTok{"black"}\NormalTok{) }\SpecialCharTok{+}  \CommentTok{\# Líneas a \# \# centroides}
  \FunctionTok{stat\_ellipse}\NormalTok{(}\FunctionTok{aes}\NormalTok{(}\AttributeTok{group =}\NormalTok{ Grupo), }\AttributeTok{linetype =} \StringTok{"dashed"}\NormalTok{) }\SpecialCharTok{+}  \CommentTok{\# Elipses por grupo}
  \FunctionTok{labs}\NormalTok{(}\AttributeTok{title =} \StringTok{"Análisis de Coordenadas Principales (PCA) con Grupos"}\NormalTok{, }
       \AttributeTok{x =} \StringTok{"Eje 1"}\NormalTok{, }\AttributeTok{y =} \StringTok{"Eje 2"}\NormalTok{) }\SpecialCharTok{+}
  \FunctionTok{theme\_minimal}\NormalTok{() }\SpecialCharTok{+}
  \FunctionTok{theme}\NormalTok{(}\AttributeTok{legend.position =} \StringTok{"right"}\NormalTok{)}
\end{Highlighting}
\end{Shaded}

\includegraphics{prueba_files/figure-latex/unnamed-chunk-26-1.pdf}

\end{document}
